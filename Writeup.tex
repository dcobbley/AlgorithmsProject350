\documentclass{article}
\usepackage[utf8]{inputenc}
\usepackage{amsmath,amssymb,amsthm,tabu,enumerate,tikz}
\usepackage[margin=1in]{geometry}
\usepackage{verbatim} % Allows Multi-line comments 
\usepackage{multicol}

\usetikzlibrary{automata,positioning}
\newcommand{\encode}[1]{\langle #1 \rangle}
  
\title{Comparison of Dijkstra's and Bellman-Ford \\ CS 350 Algorithms Project}
\author{Katie Abrahams, David Cobbley, Andrew Qin}
\date{March 12 2015}

\begin{document}

\maketitle

\section{Introduction}

We compared the Bellman-Ford and Dijkstra's shortest path algorithm on difference sets of data and compared several sets of metrics to see which algorithm is more efficient.

\section{Background info}
The Dijkstra's algorithm we are using has a time complexity of $O((|V|+|E|)+log|V|)$.

Correctness:
Completeness:

Psuedocode:
\begin{verbatim}

\end{verbatim}

Our Bellman-Ford algorithm has a time complexity of $O(|V||E|)$.

Correctness:
Completeness:

Psuedocode:
\begin{verbatim}

\end{verbatim}
\section{Testing Procedure}
We modified code for Dijktra's and Bellman-Ford algorithms found online (listed in the Sources section), and ran tests to compare the time complexity of each algorithm with controlled data sets.  We also controlled for differences in operating system efficiency; we chose Linux as our testing platform.  In the course of testing, we also used Python metrics tools (cProfile and time).
\section{Analysis}

\section{Sources}

We used this source code for our Bellman-Ford:
\begin{verbatim}
https://github.com/mneedham/algorithms2/tree/master/shortestpath
\end{verbatim}


\begin{verbatim}
http://www.ijstr.org/final-print/june2013/A-Review-And-Evaluations-Of-Shortest-Path-Algorithms.pdf

http://www.sciencedirect.com.proxy.lib.pdx.edu/science/article/pii/S0304397502006138#

http://www.sciencedirect.com.proxy.lib.pdx.edu/science/article/pii/S0196677403000464

ACM
http://dl.acm.org.proxy.lib.pdx.edu/dl.cfm
\end{verbatim}
\end{document}
